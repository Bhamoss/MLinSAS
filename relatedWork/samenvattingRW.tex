\documentclass[12pt]{article}

\usepackage[margin=1in]{geometry} 
\usepackage{amsmath,amsthm,amssymb,amsfonts}
 
\newcommand{\N}{\mathbb{N}}
\newcommand{\Z}{\mathbb{Z}}
 
\newenvironment{problem}[2][Problem]{\begin{trivlist}
\item[\hskip \labelsep {\bfseries #1}\hskip \labelsep {\bfseries #2.}]}{\end{trivlist}}
%If you want to title your bold things something different just make another thing exactly like this but replace "problem" with the name of the thing you want, like theorem or lemma or whatever
 
\begin{document}
 
%\renewcommand{\qedsymbol}{\filledbox}
%Good resources for looking up how to do stuff:
%Binary operators: http://www.access2science.com/latex/Binary.html
%General help: http://en.wikibooks.org/wiki/LaTeX/Mathematics
%Or just google stuff
 
\title{Homework template}
\author{Author}
\maketitle

\section{Legend}
Every paper will ask 4 question.
The questions are:
\begin{enumerate}

    \item What is the paper about.
    \item What is the papers contribution.
    \item Differences with our work.
    \item What we do more in our work.

\end{enumerate}

\section{Rodrigues}
\begin{enumerate}
    \item A mape-k loop is used along with UPPAAL model checking (activforms), where UPPAAL is helped by classification to create UPPAALS prediction models faster and cheaper.
    \item Faster activforms.
    \item In our work, we use classification to send a subset of adaptations to the model checker, while this work helps create models faster.
    \item Instead of speeding up model checking, we restrict its work.
\end{enumerate}
See p1 on the bottemright.

\section{Duarte}
\begin{enumerate}
    \item Combines implements mape-k with online k-plane (unsupervised learning/ clustering) to also produce humar readable output.
    \item Found a way to create non-human (= non hardcoded and flexible) knowledge but still make it human readable.
    \item No model checking and focus is on clarity instead of performance.
    \item Improve performance instead of clarity.
\end{enumerate}

\section{Schmid}
\begin{enumerate}
    \item Replaces explicit pre-build models for model checking with constructing it simpler and changing at runtime using RTX.
    \item See above.
    \item No use of mape-k or SMC, but a whole different model.
    \item Subjective because of the enormous difference.
\end{enumerate}

\section{Bencomo}
\begin{enumerate}
    \item Proposes to use bayesian networks to better deal with uncertainty in SAS.
    \item A way to adapt which is more rigid against uncertainty.
    \item While it is also uses some form of AI (I think), no model checking or explicit mape-k is mentioned.
    \item Statistical model checking
\end{enumerate}

\section{Filieri}
\begin{enumerate}
    \item The uses of dynamic binding with control theory under the hood to achieve QoS in SAS.
    \item A lightweight way to predict QoS adaptations.
    \item We use a different way to select adaptations.
    \item Model checking and ML.
\end{enumerate}

\section{Potter}
\begin{enumerate}
    \item The IoT network growing rate.
    \item A way to abstract and automatically add devices to a network using ContQuest.
    \item We focus on IoT network performance instead of structures.
    \item QoS guarantees.
\end{enumerate}

\section{Mehta}
\begin{enumerate}
    \item 5G and IoT making a demand for new strict low latency applications.
    \item Keeping latency and power and speed low with SIAA\_G and SIAA\_G\_LUR algorithms.
    \item Adapting.
    \item Using mape-k, ml and model checking.
\end{enumerate}

\section{Yang}
\begin{enumerate}
    \item Networks of entities with tasks.
    \item A learning algorithm to fullfill an entities tasks.
    \item This work does not talk about SAS, but about something resembling an IoT system. 
        They also wrote their own learning algorithm for "distributed" learning.
    \item Explicit structuring and model selection.
\end{enumerate}

\section{Nguyen}
\begin{enumerate}
    \item Non-lineair controlling systems.
    \item Applying neural network where the analytical solutions fail due to non-linearity.
    \item The problem is a task instead of maintaining a system.
    \item See above.
\end{enumerate}

\section{Ghahremani}
\begin{enumerate}
    \item Predicting models to give to MADP to use.
    \item Uses supervised learning to learn prediction models to feed to MADP with little system info.
    \item Learns the models instead of adaptations.
    \item The machine learner and model checker work seperately.
\end{enumerate}

\section{Kim}
\begin{enumerate}
    \item Trying to make SAS which does not use fixed rules to predict.
    \item Applying Reinforcement learning to select adaptation.
    \item We do not exclusively rely on machine learning.
    \item Statistical model checking as has the final say with us.
\end{enumerate}
Could be very interesting for Jaron.

\section{Andrade}
\begin{enumerate}
    \item The difficulty of modelling system to be able to use with feedback control loops.
    \item DuSE: a way to represent you system more easily along with a search algorithm to search adaptations.
    \item We work on the well established mape-k, while this paper tries to start from scratch.
    \item Improve on previous work.
\end{enumerate}

\section{Zeller}
\begin{enumerate}
    \item 2 new approaches to selecting adaptations.
    \item A way to formulate goals in transition constraints and achieving good results by applying constraint solving to it (AI). Another approach was also studied but scaled badly.
    \item We do not exclusively use AI and a different form.
    \item Combining learning with statistical model checking.
\end{enumerate}

\end{document}
